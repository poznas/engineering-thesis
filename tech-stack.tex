\chapter{Wykorzystane technologie}
\label{cha:tech-stack}

\section{Amazon Web Services}
Amazon Web Services (AWS) jest najpopularniejszym dostawcą usług cloudowych na świecie \cite{AWS-what}. 
Swoją pozycję zawdzięcza nieustannemu rozwojowi swoich produktów i wsłuchiwaniu się w potrzeby klientów.
Duże znaczenie ma również fakt, że Amazon.com jest również zbudowany na swojej platformie \cite{AS3}.
Klienci z ponad 190 krajów aktywnie korzystają z rozrastającego się wachlarza 175+ produktów, serwowanych przez centra danych położone na całym świecie \cite{AWS-O}.
AWS innowacje wprowadza już na poziomie hardware'u - projektuje własne wyspecjalizowane komponenty na użytek swoich serwerowni.
Jednym z najnowszych przykładów jest Nitro Card, 
który redefiniuje tradycyjne podejście do wirtualizacji obniżając przy tym koszty, zwiększając wydajność i bezpieczeństwo \cite{AWS-Nitro}.

\subsection{EC2}
Podstawowy produkt AWS zapewniający nam wirtualne lub {\em bare metal} instancje serwerów. 
Przychodzący i wychodzący ruch internetowy kontrolujemy przy pomocy wirtualnego firewalla - Security Group.
Poprzez mechanizm EC2 Auto Scaling możemy zdefiniować zasady ich skalowania horyzontalnego do aktualnych potrzeb.
Oferowane typy instancji różnią się dedykowanym przeznaczeniem \cite{AWS-Nitro}:

\begin{itemize}
    \item
    \textbf{M5: General Purpose}\\
    {\em Monolityczne aplikacje biznesowe}
    
    \item
    \textbf{T3: General Purpose - Burstable}\\
    {\em Mikroserwisy, interaktywne aplikacje wymagające niskiej latencji}

    \item
    \textbf{C5: Compute Optimized}\\
    {\em Aplikacje wymagające wysokiej wydajności}

    \item
    \textbf{P3dn: Machine Learning}\\
    {\em Wysoko wydajne uczenie maszynowe}

    \item
    i wiele innych
\end{itemize} 

Obraz możemy wybrać z bazowego zestawu Linux'ów i Windows'ów lub zasięgnąć AWS Marketplace, gdzie zamieszczane są obrazy budowane przez szerszą społeczność.

\subsection{DynamoDB}
Baza danych NoSQL w pełni zarządzana przez AWS. 
Zaimplementowana z myślą o wysokiej wydajności niezależnie od aktualnej skali, operacje realizowane są w jedno cyfrowej ilości milisekund.
Od użytkownika wymaga się jedynie zdefiniować strukturę tabeli. Utrzymanie i skalowanie bierze na siebie AWS \cite{AWS-O}.

\subsection{S3}
Wszechstronny serwis do przechowywania plików, używany często przez inne produkty AWS {\em (np. DynamoDB przechowuje backup'y na S3)}. 
Jednym z popularniejszych przypadków użycia jest hostowanie na S3 aplikacji webowych. 
Jak każdy z podstawowych serwisów zarządzanych przed AWS, S3 jest przygotowane na każdą skalę, zachowując przy tym wydajność i bezpieczeństwo \cite{AWS-O}.

\subsection{Route53}
Bogaty, zintegrowany z platformą serwis DNS. Najczęściej używany by nakierować ruch na instancje EC2. 
Udostępniona jest również opcja zakupu domeny bezpośrednio w konsoli webowej AWS co przyspiesza gotowość jej użycia.

\subsection{CloudFormation}
Obowiązkowe narzędzie przy pracy z większą ilością instancji produktów AWS. 
Całą swoją infrastrukturę definiuje się w czytelnym formacie pliku tekstowego {\em yaml} lub {\em json}.
Z tak przygotowanym plikiem przełożenie definicji na świat rzeczywisty jest dosłownie jednym kliknięciem. 
Równie łatwo wykonamy aktualizację lub pozbędziemy się wszystkich utworzonych produktów \cite{AWS-O}.
Taką automatyzację udostępnia również open source'owy odpowiednik tego narzędzia - Terraform.
    

