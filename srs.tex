\chapter{Analiza wymagań}
\label{cha:srs}

\section{Wymagania funkcjonalne}

\begin{figure}[!ht]
	\begin{center}
		\includegraphics[width=1\textwidth]{out/plantuml/use-cases/use-cases.pdf}
	\end{center}
    \caption{Diagram przypadków użycia \emph{thesurfadvisor.com}}
\end{figure}

Przedmiotem tej pracy jest system nowych usług dla aplikacji \emph{SurfAdvisor} - \textbf{\emph{thesurfadvisor.com}}.
Nowe rozszerzenie ma udostępniać:

\begin{itemize}
    \item
    \textbf{Logowanie za pomocą konta Google}\\
    Użytkownik ma mieć możliwość logowania poprzez swoje konto \emph{Google} tak jak w obecnej wersji aplikacji.
    Autentykacją i autoryzacją będzie dalej zarządzać istniejący serwis \emph{Firebase}, do którego nowy system \emph{thesurfadvisor.com} będzie delegował ten aspekt.

    \item
    \textbf{Przeglądanie mapy spotów}\\
    System przedstawia na mapie: 

    \begin{itemize}
    \item
    Pojedyncze spoty w formie klikalnych punktów przekierowujących

    \item
    Gęste skupiska spotów w formie nieklikalnych punktów z informacją o ich liczbie.
    Taka klasteryzacja będzie wykonywana po stronie serwerowej by nie obciążać aplikacji mobilnej.

    \end{itemize}

    \item
    \textbf{Filtrowanie spotów wyświetlanych na mapie}\\
    Zawartość mapy będzie można ograniczyć poprzez zastosowanie filtrowania po parametrach spotu.

    \item
    \textbf{Dodanie nowego spotu}\\
    Trudno jest samemu skatalogować wszystkie spoty świata, stąd system będzie polegać na treści dodawanej przez użytkowników.

    \item
    \textbf{Edycja danych spotu}\\
    Administracja \emph{SurfAdvisor} musi mieć możliwość edycji istniejących spotów w tym zmiany ich widoczności poprzez zmianę statusu.

    \item
    \textbf{Edycja literałów wyświetlanych w aplikacji mobilnej}\\
    Wszystkie labelki w nowej aplikacji mobilnej będą pobierane z repozytorium literałów.
    Pozwala to na modyfikację ich treści bez konieczności wypuszczania nowej wersji całego klienta.
\end{itemize} 


