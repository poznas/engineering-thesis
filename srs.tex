\chapter{Analiza wymagań}
\label{cha:srs}

\section{Wymagania funkcjonalne}

\begin{figure}[!ht]
	\begin{center}
		\includegraphics[width=1\textwidth]{out/plantuml/use-cases/use-cases.pdf}
	\end{center}
    \caption{Diagram przypadków użycia \emph{thesurfadvisor.com}}
\end{figure}

Przedmiotem tej pracy jest system nowych usług dla aplikacji \emph{SurfAdvisor} - \textbf{\emph{thesurfadvisor.com}}.
Nowe rozszerzenie ma udostępniać:

\begin{itemize}
    \item
    \textbf{Logowanie za pomocą konta Google}\\
    Użytkownik ma mieć możliwość logowania poprzez swoje konto \emph{Google} tak jak w obecnej wersji aplikacji.
    Autentykacją i autoryzacją będzie dalej zarządzać istniejący serwis \emph{Firebase}, do którego nowy system \emph{thesurfadvisor.com} będzie delegował ten aspekt.

    \item
    \textbf{Przeglądanie mapy spotów}\\
    System przedstawia na mapie: 

    \begin{itemize}
    \item
    Pojedyncze spoty w formie klikalnych punktów przekierowujących \\do widoku szczegółów.

    \item
    Gęste skupiska spotów w formie nieklikalnych punktów z informacją o ich liczbie.
    Taka klasteryzacja będzie wykonywana po stronie serwerowej by nie obciążać aplikacji mobilnej.

    \end{itemize}

    \item
    \textbf{Filtrowanie spotów wyświetlanych na mapie}\\
    Zawartość mapy można ograniczyć poprzez zastosowanie filtrowania po parametrach spotu.

    \item
    \textbf{Dodanie nowego spotu}\\
    Trudno jest samemu skatalogować wszystkie spoty świata, stąd system będzie polegać na treści dodawanej przez użytkowników.

    \item
    \textbf{Edycja danych spotu}\\
    Administracja \emph{SurfAdvisor} musi mieć możliwość edycji istniejących spotów w tym zmiany ich widoczności poprzez zmianę statusu.

    \item
    \textbf{Edycja literałów wyświetlanych w aplikacji mobilnej}\\
    Wszystkie labelki w nowej aplikacji mobilnej będą pobierane z repozytorium literałów.
    Pozwala to na modyfikację ich treści bez konieczności wypuszczania nowej wersji całego oprogramowania klienta.
\end{itemize} 


\section{Wymagania niefunkcjonalne}

\begin{itemize}
    \item
    \textbf{Wykorzystanie chmury}\\
    \emph{SurfAdvisor} nie posiada własnej serwerowni, dlatego nowy system powstanie na chmurze.
    Padło na AWS z uwagi na wcześniejsze doświadczenie z tą platformą autora pracy.

    \item
    \textbf{Autoskalowanie}\\
    Usługi \emph{thesurfadvisor.com} muszą się automatycznie skalować by sprostać zwiększeniom w nadchodzącym ruchu.

    \item
    \textbf{Dostępność}\\
    Przy aplikacjach przeznaczonych dla użytkowników indywidualnych nie można sobie pozwolić na przerwy w dostępności usług.
    Sczególnie w przypadku surfer'ów, którzy są aktywni również nocą często imprezując do rana.
    Aktualizacje oprogramowania serwisów muszą być przeprowadzane techniką \emph{rolling update}, która nie pozwala na choćby sekundę niedostepności.

    \item
    \textbf{Łatwe utrzymanie i rozwój}\\
    Programiści często zmieniają projekty lub pracę. Sytuacja, w której nowi pracownicy musieliby zgadywać jak dane środowisko powstało jest niedopuszczalna.
    Tutaj z pomocą przychodzi \emph{IaC} o czym więcej w sekcji \ref{iac:section}.

    \item
    \textbf{Integralność aspektu bezpieczeństwa}\\
    Autentykacja i autoryzacja będzie delegowana do istniejącego serwisu \emph{Firebase}. 
    W ten sposób w całym systemie będą używane te same identyfikatory użytkowników pomimo istnienia usług na platformach dwóch różnych dostawców chmur.

    \item
    \textbf{Monitorowanie}\\
    Parametry techniczne serwisów będą zbierane i wyświetlane w wygodnym GUI by pomóc zidentyfikować potencjalne problemy.

\end{itemize} 


