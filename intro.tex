\chapter{Wprowadzenie}
\label{cha:intro}

\section{Zawartość pracy}
\begin{enumerate}
    \item
    Pierwszy rozdział zawiera wprowadzenie wyjaśniające czym jest aplikacja mobilna \emph{SurfAdvisor} i jakie jej rozszerzenia są przedmiotem niniejszej pracy.

    \item
    W drugim rozdziale przedstawione są wymagania funkcjonalnie i niefunkcjonalne jakie nowo postwały system ma spełniać.

    \item
    Trzeci rozdział to niezbędna. Opisuje zwięźle usługi chmurowe \emph{Amazon Web Services} i orkiestracji \emph{Kubernetes} wraz z powiązanymi pojęciami.

    \item
    W czwartym rozdziale omówiono implementację - architektura pod dwoma kątami, zastosowanie \emph{IaC}, bezpieczeństwo i domenowe usługi.

    \item
    Piąty rozdział obrazuje działanie zastosowanych mechanizmów autoskalowania.

    \item
    W szóstym rozdziale zawarte zostało podsumowanie.
\end{enumerate} 


\section{Temat pracy}
pochylenie się nad tematem, jak on się objawia w tym co zdążyłem zrobić 
(o ile się da, nie przepisywałbym 1:1 tego co jest wprowadzone do systemu, bo tamta zawartość nie jest 100\% aktualna)

\section{Aplikacja obecnie}
opis czym jest SurfAdvisor na dzień dzisiejszy

\section{Cele pracy}
krótki powierzchowny opis realizowanych nowych funkcjonalności, 
skupiając się na mapie spotów (3-4 zdania) + minimalna domieszka o skalowalności/chmurze