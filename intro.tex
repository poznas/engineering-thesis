\chapter{Wprowadzenie}
\label{cha:intro}

\section{Zawartość pracy}
\begin{enumerate}
    \item
    Pierwszy rozdział zawiera wprowadzenie wyjaśniające czym jest aplikacja mobilna\\ \emph{SurfAdvisor} i jakie jej rozszerzenia są przedmiotem tej pracy.

    \item
    W drugim rozdziale przedstawione są wymagania funkcjonalne i niefunkcjonalne jakie nowo powstały system ma spełniać.

    \item
    Trzeci rozdział to niezbędna teoria. Opisuje zwięźle usługi chmurowe \emph{Amazon Web Services} i orkiestrację \emph{Kubernetes} wraz z powiązanymi pojęciami.

    \item
    W czwartym rozdziale omówiono implementację - architektura pod dwoma kątami, zastosowanie \emph{IaC}, bezpieczeństwo i domenowe usługi.

    \item
    Piąty rozdział obrazuje działanie zastosowanych mechanizmów autoskalowania.

    \item
    W szóstym rozdziale zawarte zostało podsumowanie.
\end{enumerate} 


\section{Aplikacja obecnie}
\emph{SurfAdvisor} to aplikacja mobilna na platformę \emph{Android} przeznaczona dla surfer'ów, windsurfer'ów i kitesurfer'ów.
Pozwala użytkownikom na dodawanie relacji ze spotu \emph{(miejsca, w którym uprawiają sport)} zawierającej m.in. szczegóły pogodowe.
Zbiór aktywnych relacji jest bardzo pomocny przy planowaniu surfowania. Oprócz tego w aplikacji możliwe jest przeglądanie ofert obozów i wyjazdów.
Aplikacja dostępna jest w sklepie \emph{Google Play}, twórcą jest Jan Gąsienica-Józkowy - kolega autora niniejszej pracy.
Więcej szczegółów dostępnych jest na stronie \url{http://surf-advisor.info/}

\newpage

\section{Cele pracy}
Celem pracy jest projekt i implementacja usług, które zasilą nadchodzącą rozszerzoną wersję aplikacji \emph{SurfAdvisor}.
Nastąpiła jednak zmiana biznesowej wartości jaką będzie stanowić budowany system, nie są to \emph{''oferty szkoleń i obozów''} jak w tytule.
Nowe funkcjonalności koncentrują się wokół mapy spotów, użytkownik ma mieć możliwość wygodnego jej przeglądania mając dodatkowo do dyspozycji szeroką paletę filtrów.

Tworzone usługi muszą być automatycznie skalowane tak by były w stanie obsłużyć wzmożony ruch zwłaszcza latem, gdy sporty wodne uprawiane są najintensywniej.
\emph{SurfAdvisor} nie posiada własnej serwerowni, stąd wymagane jest wykorzystanie infrastruktury w chmurze.
Nowo powstały system, zwany dalej \textbf{\emph{thesurfadvisor.com}}, ma współdziałać z istniejącymi usługami autentykacji i autoryzacji zapewnianymi przez \emph{Firebase}.

